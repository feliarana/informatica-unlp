% This is lnbip.tex the demonstration file of the LaTeX macro package for
% Lecture Notes in Business Information Processing from Springer-Verlag.
% It serves as a template for authors as well.
% version 1.0 for LaTeX2e
%
\documentclass[lnbip]{svmultln}

\usepackage[utf8]{inputenc}
\usepackage{multicol}
%
\usepackage{makeidx}  % allows for indexgeneration
% \makeindex          % be prepared for an author index
%
\begin{document}
%
\mainmatter              % start of the contribution
%
\title{Lógica e Inteligencia Artificial}
%
\titlerunning{Lógica e Inteligencia Artificial - Segunda Entrega}  % abbreviated title (for running head)
%                                     also used for the TOC unless
%                                     \toctitle is used
%
\author{Ulises J. Cornejo Fandos\inst{1}, Lucas Di Cunzolo\inst{2}, Federico Ramón Gasquez\inst{3}}
%
\authorrunning{Cornejo Fandos, Di Cunzolo, Gasquez}   % abbreviated author list (for running head)
%

%
\institute{
13566/6, Licenciatura en Informática, Facultad de Informática, UNLP
\and
13572/5, Licenciatura en Informática, Facultad de Informática, UNLP
\and
13598/6, Licenciatura en Informática, Facultad de Informática, UNLP
}

\maketitle              % typeset the title of the contribution
% \index{Ekeland, Ivar} % entries for the author index
% \index{Temam, Roger}  % of the whole volume
% \index{Dean, Jeffrey}
%

\section{Ejercicio 1}

\textit{Sean $A$, $B$ fbfs que cumplen que $(\neg A \vee B)$ es tautología. Sea C una fbf cualquiera. Determinar, si es posible, cuáles de las siguientes fbfs son tautologías y cuáles contradicciones. Justificar las respuestas.} \\

\begin{enumerate}
  \item $((\neg (A \rightarrow B)) \rightarrow C)$ \\
  
  Sean las variables de enunciados $p$ y $q$, sabemos que
  
  \begin{equation}
    (\neg p \vee q) \leftrightarrow (p \rightarrow q)
  \end{equation}
  
  es tautología. De la proposición \textit{1.10} se deduce que para formas enunciativas cualesquiera $A, B$,\
  
  \begin{equation}
    (\neg A \vee B) \leftrightarrow (A \rightarrow B)
  \end{equation}
  
  es tambien una tautología. Por lo tanto, $(\neg A \vee B)$ es logicamente equivalente a $(A \rightarrow B)$ y, reemplazando en el enunciado original, queda la siguiente formula:
  
  \begin{equation}
    ((\neg (\neg A \vee B)) \rightarrow C
    \label{eq:ejer1-1}
  \end{equation}
  
  Luego, sabemos por hipótesis que $(\neg A \vee B)$ es siempre verdadero para todo par de enunciados $A, B$, pues es una tautología. Entonces, dado que el antecedente del enunciado \ref{eq:ejer1-1}, sabemos que el mismo será \textit{falso} para todo par de enunciados $A, B$. Finalmente, independientemente del valor de verdad de la fbf $C$, sabemos que el enunciado completo será siempre verdadero, pues la unica configuración posible que permite que una implicación sea falsa es cuando el antecedente es verdadero y el consecuente falso. \\
  
  Por lo tanto, la fbf $((\neg (A \rightarrow B)) \rightarrow C)$ es una tautología. \\
  
  \item $(C \rightarrow ((\neg A) \vee B))$ \\
  
  Sean $A, B$ fbfs, sabemos que $((\neg A) \vee B)$ es una tautología por hipotesis. Luego, dado el enunciado
  
  \begin{equation}
    (C \rightarrow ((\neg A) \vee B))
  \end{equation}
  
  sabemos que el consecuente de la implicación que lo define será siempre verdadero, pues como meciona anteriormente, el mismo es una tautología. Finalmente, independientemente del valor de verdad de la fbf $C$, la implicación será siempre verdadera, pues la unica configuración posible que permite que una implicación sea falsa es cuando el antecedente es verdadero y el consecuente falso. Dado que esta configuración no tiene lugar en el enunciado, el valor de verdad es siempre verdadero. \\
  
  Por lo tanto, la fbf $(C \rightarrow ((\neg A) \vee B))$ es una tautología. \\
  
  \item $((\neg A) \rightarrow B)$ \\
  
  Sean $A, B$ fbfs, sabemos que $((\neg A) \vee B)$ es una tautología por hipotesis. Luego, dado el enunciado
  
  \begin{equation}
    ((\neg A) \rightarrow B)
    \label{eq:ejer1-3}
  \end{equation}
  
  comparamos el valor de verdad del enunciado $((\neg A) \rightarrow B) \leftrightarrow ((\neg A) \vee B)$ para verificar que se mantenga el valor de verdad. Si el mismo da una tautología, entonces el enunciado \ref{eq:ejer1-3} será una tautología. \\
  
  \textbf{HACERRRRRRRRRRRRRRRRRRRRRRRRRRRRR!} \\
\end{enumerate}

\section{Ejercicio 2}

\textit{¿Es cierto que dadas $A$ y $B$ fbfs cualesquiera, siempre ocurre que si $A$ y $A \rightarrow B$ son tautologías entonces $B$ también lo es? Fundamentar. Ejemplificar con algunos ejemplos concretos escritos en lenguaje natural.} \\

La forma de argumentación planteada se compone de dos premisas. Dadas dos fbfs cualquiera, $A, B$, la primer premisa será de la forma $A \rightarrow B$ y la segunda será $A$. Luego, siendo las premisas tautologías, 
evaluamos los casos en los que la implicación que define a la primer premisa sea verdadera. Es decir, siendo $A$ una tautología, nos quedamos con el caso en el que $B$ no haga del valor de la implicación una falsedad. \\

Por lo tanto, siendo $A$ una fbf siempre verdadera, $B$ debe ser verdadera para que $A \rightarrow B$ siga siendo una tautología y mantenga su valor de verdad. \\

Es por esto que, dadas las premisas $A$ y $A \rightarrow B$, si las mismas son tautologías, $B$ también lo es. \\ 

\textbf{\textit{Ejemplo}} \\

Sean $A$: Hoy es martes, $B$: Juan se irá a trabajar. Luego, $A \rightarrow B$: Si hoy es martes, entonces Juan se irá a trabajar. Si $A$ y $A \rightarrow B$ entonces $B$, se cumple y es válido. \\

Finalmente, este argumento es válido, pero esto no nos dice nada sobre si las premisas requeridas por el argumento son verdaderas. Para que este argumento sea un argumento sólido además de válido las premisas deberán ser verdaderas. \\

Un argumento válido pero sin solidez podría ser o no falso. El argumento de ejemplo solo es sólido los martes y cuando en efecto, se sabe que Juan realmente va a trabajar los martes. \\

\section{Ejercicio 3}



\end{document}
