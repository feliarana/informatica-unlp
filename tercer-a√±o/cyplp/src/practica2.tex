\documentclass[a4paper,10pt]{article}


\RequirePackage{color,graphicx}
\usepackage[usenames,dvipsnames]{xcolor}
\usepackage[big]{layaureo} 				%better formatting of the A4 page
% an alternative to Layaureo can be ** \usepackage{fullpage} **
\usepackage{supertabular} 				%for Grades
\usepackage{titlesec}					%custom \section
%Setup hyperref package, and colours for links
\usepackage{hyperref}
\definecolor{linkcolour}{rgb}{0,0.2,0.6}
\hypersetup{colorlinks,breaklinks,urlcolor=linkcolour, linkcolor=linkcolour}
\usepackage[utf8]{inputenc}

%Sections inspired by: 
%http://stefano.italians.nl/archives/26
\titleformat{\section}{\Large\scshape\raggedright}{}{0em}{}[\titlerule]
\titlespacing{\section}{0pt}{3pt}{20pt}
%Tweak a bit the top margin
%\addtolength{\voffset}{-1.3cm}

%Italian hyphenation for the word: ''corporations''
\hyphenation{im-pre-se}

%-------------WATERMARK TEST---------------
\usepackage[absolute]{textpos}

\setlength{\TPHorizModule}{30mm}
\setlength{\TPVertModule}{\TPHorizModule}
\textblockorigin{2mm}{0.65\paperheight}
\setlength{\parindent}{0pt}

\usepackage{vmargin}

\setpapersize{A4}
\setmargins{3.5cm}       % margen izquierdo
{2.0cm}                        % margen superior
{14.5cm}                      % anchura del texto
{24.0cm}                    % altura del texto
{10pt}                           % altura de los encabezados
{1cm}                           % espacio entre el texto y los encabezados
{0pt}                             % altura del pie de página
{2cm}                           % espacio entre el texto y el pie de página

%%%%%%%%%%%%%%%%%%%%%%%%%%%%%%%%%%%%%%%%%%%%%%%%%%%%%%%%%%%%%%%%%%
\title{Conceptos y Paradigmas de Lenguajes de Programación}
\author{Ulises J. Cornejo Fandos}
\date{Abril 2017}

\begin{document}

\maketitle

\section{Práctica 2}
    
    \begin{enumerate}
        \setcounter{enumi}{1}
        \item \textbf{¿Cuál es la importancia de la sintaxis para un lenguaje? ¿Cuáles son sus elementos?}
        
        La \textbf{sintaxis} describe la forma del lenguaje. Describe por completo la apariencia del mismo: cuáles serán sus palabras claves, cuáles serán los operadores válidos, cómo podrán ser formados los identificadores, cómo se separarán las sentencias, etc. Se encuentra estrechamente ligada a aspectos importantes de los lenguajes como la legibilidad y la facilidad de escritura, ya que una sintaxis diseñada cuidadosamente genera construcciones simples de entender y de escribir.     
        
        Los elementos de la sintaxis son:
        \begin{itemize}
            \item alfabeto
            \item identificadores
            \item operadores
            \item palabras clave y reservadas
            \item comentarios y blancos
        \end{itemize}
        
        \item \textbf{Explique, ¿a que se denomina regla lexicográfica y regla sintáctica?}
        
            Las \textbf{reglas lexicográficas} determinan, a partir del alfabeto, las \textit{words} que se usarán en el lenguaje.
    
            Las \textbf{reglas sintácticas} especifican \textit{cómo} combinar las \textit{words} para formar expresiones y sentencias.
            
        \item \textbf{¿En la definición de un lenguaje, a que se llama palabra reservada? ¿A qué son equivalentes en la definición de una gramática?}
        
            Las \textbf{palabras reservadas} son palabras que no pueden ser usadas como identificadores. Este concepto se suele confundir con el de palabra clave, que se refiere a palabras que tienen un cierto significado en un cierto contexto. En la definición de una gramática, las palabras reservadas serían G, N, T, S y P.       
        
    \end{enumerate}
\end{document}