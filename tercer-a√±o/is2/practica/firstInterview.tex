\documentclass[a4paper,10pt]{article}


\RequirePackage{color,graphicx}
\usepackage[usenames,dvipsnames]{xcolor}
\usepackage[big]{layaureo} 				%better formatting of the A4 page
% an alternative to Layaureo can be ** \usepackage{fullpage} **
\usepackage{supertabular} 				%for Grades
\usepackage{titlesec}					%custom \section
%Setup hyperref package, and colours for links
\usepackage{hyperref}
\definecolor{linkcolour}{rgb}{0,0.2,0.6}
\hypersetup{colorlinks,breaklinks,urlcolor=linkcolour, linkcolor=linkcolour}
\usepackage[utf8]{inputenc}

%Italian hyphenation for the word: ''corporations''
\hyphenation{im-pre-se}

%-------------WATERMARK TEST [**not part of a CV**]---------------
\usepackage[absolute]{textpos}

\setlength{\TPHorizModule}{30mm}
\setlength{\TPVertModule}{\TPHorizModule}
\textblockorigin{2mm}{0.65\paperheight}
\setlength{\parindent}{0pt}

\usepackage{vmargin}

\setpapersize{A4}
\setmargins{3.5cm}       % margen izquierdo
{2.0cm}                        % margen superior
{14.5cm}                      % anchura del texto
{24.0cm}                    % altura del texto
{10pt}                           % altura de los encabezados
{1cm}                           % espacio entre el texto y los encabezados
{0pt}                             % altura del pie de página
{2cm}                           % espacio entre el texto y el pie de página


%%%%%%%%%%%%%%%%%%%%%%%%%%%%%%%%%%%%%%%%%%%%%%%%%%%%%%%%%%%%%%%%%%

\title{Primer Entrevista}
\author{ COD-Project }
\date{Marzo 2017}

\begin{document}

% \maketitle


\section{Temas Abarcados}
% En esta sección se escribe cada una de las respuestas 
% luego de cada una de las preguntas con un previo salto de linea
% \\
\subsection{Blog y Favores}
\begin{enumerate}
    \item ¿Cuál es el \textbf{nombre del blog}?
    \item ¿Qué \textbf{tipo de favores} se solicitan en el blog?
    \begin{enumerate}
        \item ¿La \textbf{solicitud es paga}? $\Rightarrow$ ¿Qué medio se utiliza para efectuar el pago?
        \item ¿Deben demostrar los colaboradores su desempeño en el campo a contribuir? 
    \end{enumerate}
\end{enumerate}

\subsection{Manejo de Usuarios}
\begin{enumerate}
    \item ¿Cuenta el blog con un \textbf{registro de usuarios}?
    \begin{enumerate}
        \item ¿Este registro es \textbf{obligatorio}? 
        \item ¿Qué tipo de \textbf{información} se necesita conocer de cada usuario?
    \end{enumerate}
    \item ¿Qué funciones me permite a mi como usuario realizar el blog?
    \begin{enumerate}
        \item Se solicita diferenciar entre usuario Registrado, Logeado e invitado.
        \subitem $\hookrightarrow{}$añadir favores, comunicarme con otros usuarios, agregar amigos, eliminar favores, comentar y evaluar la atención recibida por otro usuario $\ldots$
    \end{enumerate}
\end{enumerate}

\subsection{Desarrollo de Sistema}
$\hookrightarrow{}$Se comienza a hablar del desarrollo en concreto del Sistema
\begin{enumerate}
    \item ¿Cuenta con conocimientos en el campo de informática?
    \begin{enumerate}
        \item ¿Desea aplicar algún concepto en particular en el desarrollo del sistema?
        \item ¿Ha participado en desarrollos previos de otros proyectos?
        \begin{enumerate}
            \item ¿Cuenta con documentación de los mismos? 
            \subitem $\hookrightarrow{}$ minutas de reunión, organigramas $\ldots$
        \end{enumerate}
        \item ¿El blog fue desarrollado con gestores como blogspot, wordpress, $\ldots$? 
    \end{enumerate}
    \item ¿El blog fue un desarrollo individual o realizado por alguna compañía? 
    \begin{enumerate}
        \item ¿Cuenta usted con la documentación de dicho desarrollo?
        \subitem $\hookrightarrow{}$ Esta se puede utilizar para una más clara visión del blog y el futuro sistema
    \end{enumerate}
    \item ¿Qué utilidades planea agregar con este nuevo sistema?
    \begin{enumerate}
        \item Este sistema ¿va a ser complementario al blog?
    \end{enumerate}
    \item ¿Cuales son sus objetivos con este nuevo sistema?
\end{enumerate}


\subsection{Preguntas Complementarias}
$\hookrightarrow{}$en base a respuestas del cliente en cuanto a registro de usuarios y desarrollo previo del blog se realizaran las preguntas correspondientes
\begin{enumerate}
    \item ¿Lo favores tienen un \textbf{limite de tiempo} a ser realizados?
    \begin{enumerate}
        \item Vencido este plazo, ¿son eliminados?
    \end{enumerate}
    \item ¿Se valorizan dichos favores? 
    \item ¿Puede un usuario pedir y/o brindar \textbf{más de un favor a la vez}?
    \item ¿Se \textbf{bloqueara a un usuario} que hayan recibido calificaciones negativas?
    \item ¿Puede una \textbf{empresa ser usuario}?
    \subitem $\hookrightarrow{}$ en referencia a si una organización puede solicitar o brindar algún tipo de favor
\end{enumerate}


\subsection{Preguntas Administrativas}
\begin{enumerate}
    \item ¿Es usted quien administra el blog?
    \subitem $\hookrightarrow{}$ Se solicita contacto con el administrador del blog.
    \begin{enumerate}
        \item ¿Cuenta con permisos adicionales dentro del blog?
    \end{enumerate}
    \item ¿Qué otras personas participan o intervienen en el blog?
    \begin{enumerate}
        \item ¿Estos contribuidores operan desde un establecimiento o mediante la red?
        \subitem $\hookrightarrow{}$ Distribución geográfica.
        \begin{enumerate}
            \item ¿Se puede ir a conocer y a analizar dicho establecimiento?
        \end{enumerate}
    \end{enumerate}
    \item ¿Trabajan como una organización individual o forman parte de un grupo de organizaciones?
    \item Los datos almacenados, ¿son de dominio propio o cuenta con servicios prestados?
    \subitem $\hookrightarrow{}$ Se pregunta quien administra el uso de servidores.
\end{enumerate}


\section{Formalidades y Asuntos}
$\hookrightarrow{}$ Presentación de la entrevista y la estructura de la misma. \\
$\hookrightarrow{}$ Se lleva a cabo desarrollo de preguntas. \\
$\hookrightarrow{}$ Se dialoga tiempo de desarrollo del sistema y precio del mismo. \\
$\hookrightarrow{}$ Se pide información de contacto, se pauta la siguiente reunión y contenido de la misma. \\

\end{document}
